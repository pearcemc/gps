\documentclass{article}

\usepackage{amsmath,amssymb}

\begin{document}

\section{The situation}

We want a model to account for the growth patterns of axons. It is conjectured that certain chemicals, present in a location, will encourage or discourage growth towards that area.

\section{Modelling notes}

Suppose we model the movement of an axon's tip as a particle in a two dimensional space. Then, we would expect its course over time to exhibit:

\begin{itemize}
 \item A high correlation between location at time $t$ and $t+1$ - no big jumps in location; assuming the time difference is not large.
 \item Perhaps even conditional independence between location at $t$ and at $t-2, t-3, ...$ given location at time $t-1$.
 \item Movement towards or away from the location of chemicals.
\end{itemize}

We do not get to see this axon-tip particle, however. We get a series of images. 

At each moment, the axon-tip particle lays down some axon as it goes. We could visualise this as the difference between two frames of a film: what has just been added to the image?

The structure laid down by growth of the axon might be modelled as additive over time. The visual scene at time $t$ is the sum of the visual scenes at time $t-1, t-2,...$ and the new axon accretion from time $t$ itself.

\section{A model for the axon-tip}

First, can we formalise the movement of the axon-tip particle as obeying a Gaussian process? Suppose the location of the chemical is $(c_1,c_2)^T$. 

Let's think about the $x$ co-ordinate at time $t$ being recorded at $z_i$ where $i=2t+1$ and the $y$ co-ordinate at $z_j$ where $j=2t+2$.

A possible mean function is:

\begin{flalign*}
  & \text{M}(Z_i) = \begin{cases} c_1 &\mbox{if } (-1)^{i}<0 \\ 
 c_2 & \mbox{if } (-1)^{i}>0 \end{cases}  \\
\end{flalign*}

A possible covariance kernel is:

\begin{flalign*}
  & \text{K}(Z_i,Z_j) = \begin{cases} \exp(-g||i-j||) + \sigma^2 \delta_{ij} &\mbox{if } (-1)^{i+j}>0 \\ 
 0 & \mbox{if } (-1)^{i+j}<0 \end{cases}  \\
\end{flalign*}

Where $\delta_{ij}=1$ iff $i=j$. Giving rise to the model:

\begin{flalign*}
  & Z \sim \text{GP}(M,K) \\
\end{flalign*}
 
\subsection{Using the model}

If we observe that the axon-tip begins in a particular location $\mathbf z=(z_1,z_2)^T$ then we can condition on this information.  Then the mathematics of normal distributions has it that for some other points $\mathbf f$:

\begin{flalign*}
& \left[\begin{array}{c} \mathbf z \\ \mathbf f  \end{array} \right] \sim N\left( 
\left[\begin{array}{c} M(\mathbf z) \\ M(\mathbf f)  \end{array} \right] ,
\left[\begin{array}{cc} K(\mathbf z, \mathbf z) & K(\mathbf z, \mathbf f) \\
K(\mathbf f, \mathbf z) & K(\mathbf f, \mathbf f) \end{array} \right] 
\right)
\end{flalign*}

From which the conditional distribution of the new points is:

\begin{flalign*}
& \mathbb E [\mathbf f |\mathbf z] =
 M(\mathbf f) + K(\mathbf f, \mathbf z) K(\mathbf z, \mathbf z)^{-1}(\mathbf z - M(\mathbf z)) \\
 & \text{Cov}(\mathbf f |\mathbf z) \left[\begin{array}{cc} K(\mathbf z, \mathbf z) & K(\mathbf z, \mathbf f) \\
K(\mathbf f, \mathbf z) & K(\mathbf f, \mathbf f) \end{array} \right] 
\end{flalign*}

\end{document}